% Options for packages loaded elsewhere
\PassOptionsToPackage{unicode}{hyperref}
\PassOptionsToPackage{hyphens}{url}
%
\documentclass[
]{article}
\usepackage{amsmath,amssymb}
\usepackage{iftex}
\ifPDFTeX
  \usepackage[T1]{fontenc}
  \usepackage[utf8]{inputenc}
  \usepackage{textcomp} % provide euro and other symbols
\else % if luatex or xetex
  \usepackage{unicode-math} % this also loads fontspec
  \defaultfontfeatures{Scale=MatchLowercase}
  \defaultfontfeatures[\rmfamily]{Ligatures=TeX,Scale=1}
\fi
\usepackage{lmodern}
\ifPDFTeX\else
  % xetex/luatex font selection
\fi
% Use upquote if available, for straight quotes in verbatim environments
\IfFileExists{upquote.sty}{\usepackage{upquote}}{}
\IfFileExists{microtype.sty}{% use microtype if available
  \usepackage[]{microtype}
  \UseMicrotypeSet[protrusion]{basicmath} % disable protrusion for tt fonts
}{}
\makeatletter
\@ifundefined{KOMAClassName}{% if non-KOMA class
  \IfFileExists{parskip.sty}{%
    \usepackage{parskip}
  }{% else
    \setlength{\parindent}{0pt}
    \setlength{\parskip}{6pt plus 2pt minus 1pt}}
}{% if KOMA class
  \KOMAoptions{parskip=half}}
\makeatother
\usepackage{xcolor}
\usepackage[margin=1in]{geometry}
\usepackage{color}
\usepackage{fancyvrb}
\newcommand{\VerbBar}{|}
\newcommand{\VERB}{\Verb[commandchars=\\\{\}]}
\DefineVerbatimEnvironment{Highlighting}{Verbatim}{commandchars=\\\{\}}
% Add ',fontsize=\small' for more characters per line
\usepackage{framed}
\definecolor{shadecolor}{RGB}{248,248,248}
\newenvironment{Shaded}{\begin{snugshade}}{\end{snugshade}}
\newcommand{\AlertTok}[1]{\textcolor[rgb]{0.94,0.16,0.16}{#1}}
\newcommand{\AnnotationTok}[1]{\textcolor[rgb]{0.56,0.35,0.01}{\textbf{\textit{#1}}}}
\newcommand{\AttributeTok}[1]{\textcolor[rgb]{0.13,0.29,0.53}{#1}}
\newcommand{\BaseNTok}[1]{\textcolor[rgb]{0.00,0.00,0.81}{#1}}
\newcommand{\BuiltInTok}[1]{#1}
\newcommand{\CharTok}[1]{\textcolor[rgb]{0.31,0.60,0.02}{#1}}
\newcommand{\CommentTok}[1]{\textcolor[rgb]{0.56,0.35,0.01}{\textit{#1}}}
\newcommand{\CommentVarTok}[1]{\textcolor[rgb]{0.56,0.35,0.01}{\textbf{\textit{#1}}}}
\newcommand{\ConstantTok}[1]{\textcolor[rgb]{0.56,0.35,0.01}{#1}}
\newcommand{\ControlFlowTok}[1]{\textcolor[rgb]{0.13,0.29,0.53}{\textbf{#1}}}
\newcommand{\DataTypeTok}[1]{\textcolor[rgb]{0.13,0.29,0.53}{#1}}
\newcommand{\DecValTok}[1]{\textcolor[rgb]{0.00,0.00,0.81}{#1}}
\newcommand{\DocumentationTok}[1]{\textcolor[rgb]{0.56,0.35,0.01}{\textbf{\textit{#1}}}}
\newcommand{\ErrorTok}[1]{\textcolor[rgb]{0.64,0.00,0.00}{\textbf{#1}}}
\newcommand{\ExtensionTok}[1]{#1}
\newcommand{\FloatTok}[1]{\textcolor[rgb]{0.00,0.00,0.81}{#1}}
\newcommand{\FunctionTok}[1]{\textcolor[rgb]{0.13,0.29,0.53}{\textbf{#1}}}
\newcommand{\ImportTok}[1]{#1}
\newcommand{\InformationTok}[1]{\textcolor[rgb]{0.56,0.35,0.01}{\textbf{\textit{#1}}}}
\newcommand{\KeywordTok}[1]{\textcolor[rgb]{0.13,0.29,0.53}{\textbf{#1}}}
\newcommand{\NormalTok}[1]{#1}
\newcommand{\OperatorTok}[1]{\textcolor[rgb]{0.81,0.36,0.00}{\textbf{#1}}}
\newcommand{\OtherTok}[1]{\textcolor[rgb]{0.56,0.35,0.01}{#1}}
\newcommand{\PreprocessorTok}[1]{\textcolor[rgb]{0.56,0.35,0.01}{\textit{#1}}}
\newcommand{\RegionMarkerTok}[1]{#1}
\newcommand{\SpecialCharTok}[1]{\textcolor[rgb]{0.81,0.36,0.00}{\textbf{#1}}}
\newcommand{\SpecialStringTok}[1]{\textcolor[rgb]{0.31,0.60,0.02}{#1}}
\newcommand{\StringTok}[1]{\textcolor[rgb]{0.31,0.60,0.02}{#1}}
\newcommand{\VariableTok}[1]{\textcolor[rgb]{0.00,0.00,0.00}{#1}}
\newcommand{\VerbatimStringTok}[1]{\textcolor[rgb]{0.31,0.60,0.02}{#1}}
\newcommand{\WarningTok}[1]{\textcolor[rgb]{0.56,0.35,0.01}{\textbf{\textit{#1}}}}
\usepackage{graphicx}
\makeatletter
\def\maxwidth{\ifdim\Gin@nat@width>\linewidth\linewidth\else\Gin@nat@width\fi}
\def\maxheight{\ifdim\Gin@nat@height>\textheight\textheight\else\Gin@nat@height\fi}
\makeatother
% Scale images if necessary, so that they will not overflow the page
% margins by default, and it is still possible to overwrite the defaults
% using explicit options in \includegraphics[width, height, ...]{}
\setkeys{Gin}{width=\maxwidth,height=\maxheight,keepaspectratio}
% Set default figure placement to htbp
\makeatletter
\def\fps@figure{htbp}
\makeatother
\setlength{\emergencystretch}{3em} % prevent overfull lines
\providecommand{\tightlist}{%
  \setlength{\itemsep}{0pt}\setlength{\parskip}{0pt}}
\setcounter{secnumdepth}{-\maxdimen} % remove section numbering
\ifLuaTeX
  \usepackage{selnolig}  % disable illegal ligatures
\fi
\IfFileExists{bookmark.sty}{\usepackage{bookmark}}{\usepackage{hyperref}}
\IfFileExists{xurl.sty}{\usepackage{xurl}}{} % add URL line breaks if available
\urlstyle{same}
\hypersetup{
  pdftitle={ACP\_University},
  pdfauthor={Moussa},
  hidelinks,
  pdfcreator={LaTeX via pandoc}}

\title{ACP\_University}
\author{Moussa}
\date{2024-08-18}

\begin{document}
\maketitle

\hypertarget{import-of-dataset}{%
\subparagraph{Import of dataset}\label{import-of-dataset}}

\begin{Shaded}
\begin{Highlighting}[]
\FunctionTok{library}\NormalTok{(readxl)}
\NormalTok{universite }\OtherTok{\textless{}{-}} \FunctionTok{read\_excel}\NormalTok{(}\StringTok{"\textasciitilde{}/GitHub/SORADATA/ACP{-}R/Dataset/universite.xlsx"}\NormalTok{)}
\end{Highlighting}
\end{Shaded}

\begin{verbatim}
## New names:
## * `` -> `...1`
\end{verbatim}

\begin{Shaded}
\begin{Highlighting}[]
\FunctionTok{str}\NormalTok{(universite)}
\end{Highlighting}
\end{Shaded}

\begin{verbatim}
## tibble [10 x 13] (S3: tbl_df/tbl/data.frame)
##  $ ...1      : chr [1:10] "Droit, sciences politiques" "Sciences économiques, gestion" "Administration économique et sociale" "Lettres, sciences du langage, arts" ...
##  $ Licence-F : num [1:10] 69373 38387 18574 48691 62736 ...
##  $ Licence-H : num [1:10] 37317 37157 12388 17850 21291 ...
##  $ Master-F  : num [1:10] 42371 29466 4183 17672 13186 ...
##  $ Master-H  : num [1:10] 21693 26929 2884 5853 3874 ...
##  $ Doctorat-F: num [1:10] 4029 1983 0 4531 1839 ...
##  $ Doctorat-H: num [1:10] 4342 2552 0 2401 907 ...
##  $ Total-F   : num [1:10] 115773 69836 22757 70894 77761 ...
##  $ Total-H   : num [1:10] 63352 66638 15272 26104 26072 ...
##  $ Licence   : num [1:10] 106690 75544 30962 66541 84027 ...
##  $ Master    : num [1:10] 64064 56395 7067 23525 17060 ...
##  $ Doctorat  : num [1:10] 8371 4535 0 6932 2746 ...
##  $ Total     : num [1:10] 179125 136474 38029 96998 103833 ...
\end{verbatim}

\hypertarget{les-statistiques-descriptives}{%
\paragraph{Les statistiques
descriptives}\label{les-statistiques-descriptives}}

\begin{Shaded}
\begin{Highlighting}[]
\FunctionTok{summary}\NormalTok{(universite)}
\end{Highlighting}
\end{Shaded}

\begin{verbatim}
##      ...1             Licence-F       Licence-H        Master-F    
##  Length:10          Min.   : 1779   Min.   :  726   Min.   : 1963  
##  Class :character   1st Qu.:19570   1st Qu.:15566   1st Qu.: 5910  
##  Mode  :character   Median :31353   Median :19571   Median :15132  
##                     Mean   :38901   Mean   :25490   Mean   :18238  
##                     3rd Qu.:59225   3rd Qu.:37277   3rd Qu.:26518  
##                     Max.   :94346   Max.   :54861   Max.   :43016  
##     Master-H       Doctorat-F       Doctorat-H         Total-F      
##  Min.   :  811   Min.   :   0.0   Min.   :    0.0   Min.   :  4148  
##  1st Qu.: 3948   1st Qu.: 600.8   1st Qu.:  472.8   1st Qu.: 27330  
##  Median : 7155   Median :3006.0   Median : 2476.5   Median : 56940  
##  Mean   :14341   Mean   :3041.8   Mean   : 3424.0   Mean   : 60181  
##  3rd Qu.:21382   3rd Qu.:4500.0   3rd Qu.: 5009.5   3rd Qu.: 76044  
##  Max.   :48293   Max.   :7787.0   Max.   :11491.0   Max.   :145149  
##     Total-H          Licence           Master         Doctorat    
##  Min.   :  1552   Min.   :  2505   Min.   : 3167   Min.   :    0  
##  1st Qu.: 22833   1st Qu.: 33052   1st Qu.: 9565   1st Qu.: 1074  
##  Median : 27399   Median : 71043   Median :21536   Median : 5734  
##  Mean   : 43255   Mean   : 64391   Mean   :32579   Mean   : 6466  
##  3rd Qu.: 65817   3rd Qu.: 82375   3rd Qu.:61696   3rd Qu.:10248  
##  Max.   :114645   Max.   :135396   Max.   :65371   Max.   :15898  
##      Total       
##  Min.   :  5700  
##  1st Qu.: 45957  
##  Median :100416  
##  Mean   :103436  
##  3rd Qu.:153135  
##  Max.   :213618
\end{verbatim}

\begin{Shaded}
\begin{Highlighting}[]
\FunctionTok{aggregate}\NormalTok{(Master}\SpecialCharTok{\textasciitilde{}}\NormalTok{Doctorat, }\AttributeTok{data =}\NormalTok{ universite, }\AttributeTok{FUN =}\NormalTok{ mean)}
\end{Highlighting}
\end{Shaded}

\begin{verbatim}
##    Doctorat Master
## 1         0   7067
## 2        28   3167
## 3       516   6135
## 4      2746  17060
## 5      4535  56395
## 6      6932  23525
## 7      8371  64064
## 8     10873  19547
## 9     14759  63463
## 10    15898  65371
\end{verbatim}

\begin{Shaded}
\begin{Highlighting}[]
\FunctionTok{library}\NormalTok{(ggplot2)}

\CommentTok{\# Préparer les données pour ggplot}
\NormalTok{universite\_long }\OtherTok{\textless{}{-}}\NormalTok{ tidyr}\SpecialCharTok{::}\FunctionTok{pivot\_longer}\NormalTok{(universite, }\AttributeTok{cols =} \FunctionTok{c}\NormalTok{(}\StringTok{\textasciigrave{}}\AttributeTok{Doctorat{-}F}\StringTok{\textasciigrave{}}\NormalTok{, }\StringTok{\textasciigrave{}}\AttributeTok{Doctorat{-}H}\StringTok{\textasciigrave{}}\NormalTok{, }\StringTok{\textasciigrave{}}\AttributeTok{Master{-}F}\StringTok{\textasciigrave{}}\NormalTok{, }\StringTok{\textasciigrave{}}\AttributeTok{Master{-}H}\StringTok{\textasciigrave{}}\NormalTok{), }
                                       \AttributeTok{names\_to =} \FunctionTok{c}\NormalTok{(}\StringTok{"Niveau"}\NormalTok{, }\StringTok{"Sexe"}\NormalTok{), }\AttributeTok{names\_sep =} \StringTok{"{-}"}\NormalTok{, }\AttributeTok{values\_to =} \StringTok{"Valeurs"}\NormalTok{)}

\CommentTok{\# Créer l\textquotesingle{}histogramme avec ggplot pour Doctorat et Master}
\FunctionTok{ggplot}\NormalTok{(universite\_long, }\FunctionTok{aes}\NormalTok{(}\AttributeTok{x =}\NormalTok{ Valeurs, }\AttributeTok{fill =}\NormalTok{ Sexe)) }\SpecialCharTok{+} 
  \FunctionTok{geom\_histogram}\NormalTok{(}\AttributeTok{alpha =} \FloatTok{0.5}\NormalTok{, }\AttributeTok{position =} \StringTok{"identity"}\NormalTok{) }\SpecialCharTok{+}
  \FunctionTok{facet\_wrap}\NormalTok{(}\SpecialCharTok{\textasciitilde{}}\NormalTok{Niveau) }\SpecialCharTok{+}
  \FunctionTok{labs}\NormalTok{(}\AttributeTok{title =} \StringTok{"Comparaison Doctorat et Master par Sexe"}\NormalTok{, }\AttributeTok{x =} \StringTok{"Valeurs"}\NormalTok{, }\AttributeTok{y =} \StringTok{"Fréquence"}\NormalTok{) }\SpecialCharTok{+}
  \FunctionTok{scale\_fill\_manual}\NormalTok{(}\AttributeTok{values =} \FunctionTok{c}\NormalTok{(}\StringTok{"blue"}\NormalTok{, }\StringTok{"red"}\NormalTok{, }\StringTok{"green"}\NormalTok{, }\StringTok{"orange"}\NormalTok{))}
\end{Highlighting}
\end{Shaded}

\begin{verbatim}
## `stat_bin()` using `bins = 30`. Pick better value with `binwidth`.
\end{verbatim}

\includegraphics{ACP_Univ_files/figure-latex/unnamed-chunk-5-1.pdf}

\begin{Shaded}
\begin{Highlighting}[]
\FunctionTok{library}\NormalTok{(ggplot2)}

\CommentTok{\# Préparer les données au format long}
\NormalTok{universite\_long }\OtherTok{\textless{}{-}}\NormalTok{ tidyr}\SpecialCharTok{::}\FunctionTok{pivot\_longer}\NormalTok{(universite, }\AttributeTok{cols =} \FunctionTok{c}\NormalTok{(}\StringTok{\textasciigrave{}}\AttributeTok{Doctorat{-}F}\StringTok{\textasciigrave{}}\NormalTok{, }\StringTok{\textasciigrave{}}\AttributeTok{Doctorat{-}H}\StringTok{\textasciigrave{}}\NormalTok{, }\StringTok{\textasciigrave{}}\AttributeTok{Master{-}F}\StringTok{\textasciigrave{}}\NormalTok{, }\StringTok{\textasciigrave{}}\AttributeTok{Master{-}H}\StringTok{\textasciigrave{}}\NormalTok{), }
                                       \AttributeTok{names\_to =} \FunctionTok{c}\NormalTok{(}\StringTok{"Niveau"}\NormalTok{, }\StringTok{"Sexe"}\NormalTok{), }\AttributeTok{names\_sep =} \StringTok{"{-}"}\NormalTok{, }\AttributeTok{values\_to =} \StringTok{"Valeurs"}\NormalTok{)}

\CommentTok{\# Créer le boxplot avec ggplot}
\FunctionTok{ggplot}\NormalTok{(universite\_long, }\FunctionTok{aes}\NormalTok{(}\AttributeTok{x =} \FunctionTok{interaction}\NormalTok{(Niveau, Sexe), }\AttributeTok{y =}\NormalTok{ Valeurs, }\AttributeTok{fill =}\NormalTok{ Sexe)) }\SpecialCharTok{+}
  \FunctionTok{geom\_boxplot}\NormalTok{() }\SpecialCharTok{+}
  \FunctionTok{labs}\NormalTok{(}\AttributeTok{title =} \StringTok{"Boxplot Doctorat et Master par Sexe"}\NormalTok{, }\AttributeTok{x =} \StringTok{"Niveau et Sexe"}\NormalTok{, }\AttributeTok{y =} \StringTok{"Valeurs"}\NormalTok{) }\SpecialCharTok{+}
  \FunctionTok{scale\_fill\_manual}\NormalTok{(}\AttributeTok{values =} \FunctionTok{c}\NormalTok{(}\StringTok{"lightblue"}\NormalTok{, }\StringTok{"lightpink"}\NormalTok{)) }\SpecialCharTok{+}
  \FunctionTok{theme\_minimal}\NormalTok{()}
\end{Highlighting}
\end{Shaded}

\includegraphics{ACP_Univ_files/figure-latex/unnamed-chunk-6-1.pdf}

\begin{Shaded}
\begin{Highlighting}[]
\CommentTok{\# Calculer le total pour chaque filière}
\NormalTok{total\_doctorat }\OtherTok{\textless{}{-}} \FunctionTok{sum}\NormalTok{(universite}\SpecialCharTok{$}\StringTok{\textasciigrave{}}\AttributeTok{Doctorat{-}F}\StringTok{\textasciigrave{}}\NormalTok{) }\SpecialCharTok{+} \FunctionTok{sum}\NormalTok{(universite}\SpecialCharTok{$}\StringTok{\textasciigrave{}}\AttributeTok{Doctorat{-}H}\StringTok{\textasciigrave{}}\NormalTok{)}
\NormalTok{total\_master }\OtherTok{\textless{}{-}} \FunctionTok{sum}\NormalTok{(universite}\SpecialCharTok{$}\StringTok{\textasciigrave{}}\AttributeTok{Master{-}F}\StringTok{\textasciigrave{}}\NormalTok{) }\SpecialCharTok{+} \FunctionTok{sum}\NormalTok{(universite}\SpecialCharTok{$}\StringTok{\textasciigrave{}}\AttributeTok{Master{-}H}\StringTok{\textasciigrave{}}\NormalTok{)}

\CommentTok{\# Calculer le total général}
\NormalTok{total\_global }\OtherTok{\textless{}{-}}\NormalTok{ total\_doctorat }\SpecialCharTok{+}\NormalTok{ total\_master}

\CommentTok{\# Calculer les proportions}
\NormalTok{proportion\_doctorat }\OtherTok{\textless{}{-}}\NormalTok{ total\_doctorat }\SpecialCharTok{/}\NormalTok{ total\_global}
\NormalTok{proportion\_master }\OtherTok{\textless{}{-}}\NormalTok{ total\_master }\SpecialCharTok{/}\NormalTok{ total\_global}

\CommentTok{\# Créer un vecteur des proportions}
\NormalTok{proportions }\OtherTok{\textless{}{-}} \FunctionTok{c}\NormalTok{(proportion\_doctorat, proportion\_master)}

\CommentTok{\# Créer un vecteur des noms des filières}
\NormalTok{filieres }\OtherTok{\textless{}{-}} \FunctionTok{c}\NormalTok{(}\StringTok{"Doctorat"}\NormalTok{, }\StringTok{"Master"}\NormalTok{)}
\FunctionTok{library}\NormalTok{(ggplot2)}

\CommentTok{\# Préparer les données dans un data frame}
\NormalTok{proportions\_df }\OtherTok{\textless{}{-}} \FunctionTok{data.frame}\NormalTok{(}
  \AttributeTok{Filiere =}\NormalTok{ filieres,}
  \AttributeTok{Proportion =}\NormalTok{ proportions}
\NormalTok{)}

\CommentTok{\# Créer le barplot avec ggplot2}
\FunctionTok{ggplot}\NormalTok{(proportions\_df, }\FunctionTok{aes}\NormalTok{(}\AttributeTok{x =}\NormalTok{ Filiere, }\AttributeTok{y =}\NormalTok{ Proportion, }\AttributeTok{fill =}\NormalTok{ Filiere)) }\SpecialCharTok{+}
  \FunctionTok{geom\_bar}\NormalTok{(}\AttributeTok{stat =} \StringTok{"identity"}\NormalTok{) }\SpecialCharTok{+}
  \FunctionTok{scale\_y\_continuous}\NormalTok{(}\AttributeTok{labels =}\NormalTok{ scales}\SpecialCharTok{::}\NormalTok{percent, }\AttributeTok{limits =} \FunctionTok{c}\NormalTok{(}\DecValTok{0}\NormalTok{, }\DecValTok{1}\NormalTok{)) }\SpecialCharTok{+}
  \FunctionTok{labs}\NormalTok{(}\AttributeTok{title =} \StringTok{"Proportions des Étudiants par niveau d\textquotesingle{}étude"}\NormalTok{, }\AttributeTok{x =} \StringTok{"Niveau d\textquotesingle{}étude"}\NormalTok{, }\AttributeTok{y =} \StringTok{"Proportion"}\NormalTok{) }\SpecialCharTok{+}
  \FunctionTok{scale\_fill\_manual}\NormalTok{(}\AttributeTok{values =} \FunctionTok{c}\NormalTok{(}\StringTok{"lightblue"}\NormalTok{, }\StringTok{"lightgreen"}\NormalTok{)) }\SpecialCharTok{+}
  \FunctionTok{theme\_minimal}\NormalTok{() }\SpecialCharTok{+}
  \FunctionTok{geom\_text}\NormalTok{(}\FunctionTok{aes}\NormalTok{(}\AttributeTok{label =}\NormalTok{ scales}\SpecialCharTok{::}\FunctionTok{percent}\NormalTok{(Proportion)), }\AttributeTok{vjust =} \SpecialCharTok{{-}}\FloatTok{0.5}\NormalTok{)}
\end{Highlighting}
\end{Shaded}

\includegraphics{ACP_Univ_files/figure-latex/unnamed-chunk-7-1.pdf}

\hypertarget{acp-via-factominer}{%
\paragraph{ACP via FactomineR}\label{acp-via-factominer}}

\begin{Shaded}
\begin{Highlighting}[]
\FunctionTok{library}\NormalTok{(FactoMineR)}
\FunctionTok{library}\NormalTok{(}\StringTok{"factoextra"}\NormalTok{)}
\end{Highlighting}
\end{Shaded}

\begin{verbatim}
## Welcome! Want to learn more? See two factoextra-related books at https://goo.gl/ve3WBa
\end{verbatim}

\begin{Shaded}
\begin{Highlighting}[]
\CommentTok{\# Sélectionner uniquement les colonnes numériques}
\NormalTok{numerical\_vars }\OtherTok{\textless{}{-}} \FunctionTok{sapply}\NormalTok{(universite, is.numeric)}

\CommentTok{\# Créer un sous{-}data frame avec uniquement les colonnes numériques}
\NormalTok{universite\_numeric }\OtherTok{\textless{}{-}}\NormalTok{ universite[, numerical\_vars]}
\end{Highlighting}
\end{Shaded}

\begin{Shaded}
\begin{Highlighting}[]
\NormalTok{resultat\_ACP}\OtherTok{\textless{}{-}}\FunctionTok{PCA}\NormalTok{(universite\_numeric, }\AttributeTok{graph =} \ConstantTok{FALSE}\NormalTok{)}
\FunctionTok{print}\NormalTok{(resultat\_ACP)}
\end{Highlighting}
\end{Shaded}

\begin{verbatim}
## **Results for the Principal Component Analysis (PCA)**
## The analysis was performed on 10 individuals, described by 12 variables
## *The results are available in the following objects:
## 
##    name               description                          
## 1  "$eig"             "eigenvalues"                        
## 2  "$var"             "results for the variables"          
## 3  "$var$coord"       "coord. for the variables"           
## 4  "$var$cor"         "correlations variables - dimensions"
## 5  "$var$cos2"        "cos2 for the variables"             
## 6  "$var$contrib"     "contributions of the variables"     
## 7  "$ind"             "results for the individuals"        
## 8  "$ind$coord"       "coord. for the individuals"         
## 9  "$ind$cos2"        "cos2 for the individuals"           
## 10 "$ind$contrib"     "contributions of the individuals"   
## 11 "$call"            "summary statistics"                 
## 12 "$call$centre"     "mean of the variables"              
## 13 "$call$ecart.type" "standard error of the variables"    
## 14 "$call$row.w"      "weights for the individuals"        
## 15 "$call$col.w"      "weights for the variables"
\end{verbatim}

\hypertarget{le-choix-de-laxe-ou-de-dimension}{%
\subsubsection{Le choix de l'axe ou de
dimension}\label{le-choix-de-laxe-ou-de-dimension}}

\begin{Shaded}
\begin{Highlighting}[]
\NormalTok{valeurspropes}\OtherTok{\textless{}{-}}\NormalTok{resultat\_ACP}\SpecialCharTok{$}\NormalTok{eig}
\NormalTok{valeurspropes}
\end{Highlighting}
\end{Shaded}

\begin{verbatim}
##          eigenvalue percentage of variance cumulative percentage of variance
## comp 1 9.008252e+00           7.506876e+01                          75.06876
## comp 2 1.988651e+00           1.657209e+01                          91.64086
## comp 3 8.032154e-01           6.693461e+00                          98.33432
## comp 4 1.680335e-01           1.400279e+00                          99.73460
## comp 5 2.304292e-02           1.920243e-01                          99.92662
## comp 6 8.805451e-03           7.337875e-02                         100.00000
## comp 7 1.985892e-31           1.654910e-30                         100.00000
## comp 8 1.112184e-32           9.268203e-32                         100.00000
## comp 9 2.901728e-33           2.418106e-32                         100.00000
\end{verbatim}

\begin{Shaded}
\begin{Highlighting}[]
\FunctionTok{barplot}\NormalTok{(valeurspropes[,}\DecValTok{2}\NormalTok{],}\AttributeTok{names.arg =} \DecValTok{1}\SpecialCharTok{:}\FunctionTok{nrow}\NormalTok{(valeurspropes),}
        \AttributeTok{main =} \StringTok{"Pourcentage dela variance expliquée par chaque composante"}\NormalTok{,}
        \AttributeTok{xlab =} \StringTok{"Composantes principales"}\NormalTok{,}
        \AttributeTok{ylab =} \StringTok{"Pourcentage de variance expliquée"}\NormalTok{,}
        \AttributeTok{col =} \StringTok{"steelblue"}\NormalTok{)}
\FunctionTok{lines}\NormalTok{(}\AttributeTok{x=}\DecValTok{1}\SpecialCharTok{:}\FunctionTok{nrow}\NormalTok{(valeurspropes),valeurspropes[,}\DecValTok{2}\NormalTok{],}
      \AttributeTok{type =} \StringTok{"b"}\NormalTok{,}\AttributeTok{pch=}\DecValTok{19}\NormalTok{,}\AttributeTok{col=}\StringTok{"red"}\NormalTok{)}
\end{Highlighting}
\end{Shaded}

\includegraphics{ACP_Univ_files/figure-latex/unnamed-chunk-12-1.pdf}

\hypertarget{le-cercle-de-corruxe9lation-ou-il-exite-un-effet-de-taille}{%
\subsubsection{Le cercle de corrélation ou il exite un effet de
taille}\label{le-cercle-de-corruxe9lation-ou-il-exite-un-effet-de-taille}}

\begin{Shaded}
\begin{Highlighting}[]
\FunctionTok{fviz\_pca\_var}\NormalTok{(resultat\_ACP,}
             \AttributeTok{col.var =} \StringTok{"cos2"}\NormalTok{, }\CommentTok{\# Coloration par le cosinus carré}
             \AttributeTok{gradient.cols =} \FunctionTok{c}\NormalTok{(}\StringTok{"\#888AFB"}\NormalTok{,}\StringTok{"\#E78800"}\NormalTok{,}\StringTok{"\#FC4E07"}\NormalTok{), }\CommentTok{\# Définir les couleurs du gradient}
             \AttributeTok{repel =} \ConstantTok{TRUE}\NormalTok{, }\CommentTok{\# Éviter le chevauchement des étiquettes}
             \AttributeTok{title =} \StringTok{"Cercle de corrélation des variables"}\NormalTok{) }\CommentTok{\# Titre du graphique}
\end{Highlighting}
\end{Shaded}

\includegraphics{ACP_Univ_files/figure-latex/unnamed-chunk-13-1.pdf}

\begin{Shaded}
\begin{Highlighting}[]
\FunctionTok{fviz\_pca\_ind}\NormalTok{(resultat\_ACP, }
             \AttributeTok{col.ind =} \StringTok{"cos2"}\NormalTok{,  }\CommentTok{\# Coloration en fonction du cos2}
             \AttributeTok{gradient.cols =} \FunctionTok{c}\NormalTok{(}\StringTok{"blue"}\NormalTok{, }\StringTok{"white"}\NormalTok{, }\StringTok{"red"}\NormalTok{),  }\CommentTok{\# Palette de couleurs}
             \AttributeTok{repel =} \ConstantTok{TRUE}\NormalTok{) }\SpecialCharTok{+}  \CommentTok{\# Éviter le chevauchement des étiquettes}
  \FunctionTok{scale\_color\_gradient2}\NormalTok{(}\AttributeTok{low =} \StringTok{"blue"}\NormalTok{, }\AttributeTok{mid =} \StringTok{"white"}\NormalTok{, }\AttributeTok{high =} \StringTok{"red"}\NormalTok{, }\AttributeTok{midpoint =} \FloatTok{0.50}\NormalTok{) }\SpecialCharTok{+}
  \FunctionTok{theme\_minimal}\NormalTok{() }\SpecialCharTok{+}
  \FunctionTok{labs}\NormalTok{(}\AttributeTok{title =} \StringTok{"Visualisation des individus selon Cos2"}\NormalTok{,}
       \AttributeTok{color =} \StringTok{"Cos2"}\NormalTok{)}
\end{Highlighting}
\end{Shaded}

\begin{verbatim}
## Scale for colour is already present.
## Adding another scale for colour, which will replace the existing scale.
\end{verbatim}

\includegraphics{ACP_Univ_files/figure-latex/unnamed-chunk-14-1.pdf}

\end{document}
